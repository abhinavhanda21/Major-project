\section{Software Requirement Specification}
The main requirement is to some how paralleize the evaluation of nodes in the node tree. This can be achieved by either taking a multithreading approach or having two separate processes for compile and render itself. The major concern is the non thread safety of CGAL library which needs to be accounted for.
Another follow up requirement is to add a cancel mechnism to any ongoing render process without loosing context on already evaluated nodes in the tree.

\subsection{Functional Requirements}
\begin{itemize}
	\item Parallel processing of Nodes.
	\item Maintaining thread safety of CGAL.
	\item Sustaining caching of previosuly evaluated nodes to improve speed during repeated renders of the model.
	\item Introducing a halt mechanism on ongoing render processes.
	\item Safe exit from compilation.
	\item Warning in case of multiple pseudo root node tags.
	\item Error detection of unclosed multiple comments, use and include tags and double quotes used for representing strings.
\end{itemize}

\subsection{Performance Requirement}
\begin{itemize}
	\item Improvement in time consumption during rendering.
	\item Efficient use of processor resources.
	\item Less CPU idle time.
\end{itemize}

\subsection{Dependability Requirement}
\begin{itemize}
	\item All previously supported OS must also be supported after additional changes and features.
	\item libraries used must be cross platform in nature.
	\item Thread safety must be maintained across all operating system. The interface for threads used must OS independant.
\end{itemize}
\subsection{Non-functional requirements}
\begin{enumerate}
	\item Extensible: It should be able to support future functional requirements
	\item Usability: Simple user interfaces that a layman can understand.
	\item Modular Structure: The software should have  structure. So, that different parts of software would be changed without affecting other parts.
	\item Backward compatibility: Addition of new syntax should not forbid script to work correctly on the backward versions of OpenSCAD.
\end{enumerate}

